
\documentclass{article}
\usepackage[UTF8]{ctex}
% Replace `letterpaper' with`a4paper' for UK/EU standard size
\usepackage[a4paper,top=2cm,bottom=2cm,left=2cm,right=2cm,marginparwidth=1.75cm]{geometry}

% Useful packages
\usepackage{amsmath}
\usepackage{graphicx}
\usepackage[colorlinks=true, allcolors=blue]{hyperref}
\usepackage{graphicx} %插入图片的宏包
\usepackage{float} %设置图片浮动位置的宏包
\usepackage{subfigure} %插入多图时用子图显示的宏包
\usepackage{parskip}
\usepackage{indentfirst} 
\setlength{\parindent}{2em}
\usepackage{hyperref}  
\usepackage{tikz}
\allowdisplaybreaks
\usepackage{multirow}
\usepackage{amsmath}
\usepackage{bookmark}
\usepackage{amsfonts,amssymb} 
\usepackage{xcolor} % 用于显示颜色
\usepackage{listings} % 用于插入代码
\usepackage{tikz} % 导入多重比较示意图专用
\usetikzlibrary{decorations,arrows,shapes} % 导入多重比较示意图专用

\lstset{
	basicstyle          =   \sffamily,          % 基本代码风格
	keywordstyle        =   \bfseries,          % 关键字风格
	commentstyle        =   \rmfamily\itshape,  % 注释的风格,斜体
	stringstyle         =   \ttfamily,  % 字符串风格
	flexiblecolumns,                % 别问为什么,加上这个
	numbers             =   left,   % 行号的位置在左边
	showspaces          =   false,  % 是否显示空格,显示了有点乱,所以不现实了
	numberstyle         =   \zihao{-5}\ttfamily,    % 行号的样式,小五号,tt等宽字体
	showstringspaces    =   false,
	captionpos          =   t,      % 这段代码的名字所呈现的位置,t指的是top上面
	frame               =   lrtb,   % 显示边框
}

\lstdefinestyle{R}{
	language        =   R, % 语言选R
	basicstyle      =   \zihao{-5}\ttfamily,
	numberstyle     =   \zihao{-5}\ttfamily,
	keywordstyle    =   \color{blue},
	keywordstyle    =   [2] \color{teal},
	stringstyle     =   \color{magenta},
	commentstyle    =   \color{red}\ttfamily,
	breaklines      =   true,   % 自动换行,建议不要写太长的行
	columns         =   fixed,  % 如果不加这一句,字间距就不固定,很丑,必须加
	basewidth       =   0.5em,
}

\lstdefinestyle{Python}{
	language        =   Python, % 语言选R
	basicstyle      =   \zihao{-5}\ttfamily,
	numberstyle     =   \zihao{-5}\ttfamily,
	keywordstyle    =   \color{blue},
	keywordstyle    =   [2] \color{teal},
	stringstyle     =   \color{magenta},
	commentstyle    =   \color{red}\ttfamily,
	breaklines      =   true,   % 自动换行,建议不要写太长的行
	columns         =   fixed,  % 如果不加这一句,字间距就不固定,很丑,必须加
	basewidth       =   0.5em,
}

\title{第三问和第四问补充部分}
\author{}
\date{}

\begin{document}
	\maketitle
	% \tableofcontents

% Please add the following required packages to your document preamble:
% \usepackage{multirow}


\begin{align*}
    \log(p) = &\beta_0 + \beta_1 \times Treat + \beta_2 \times Weight \,+ \\
	&\beta_3 \times Europe + \beta_4 \times America + \beta_5 \times age
\end{align*}






\begin{table}[H]
	\centering
	\caption{以log为连接函数的广义线性方程回归结果}
\begin{tabular}{lcccl}
	\hline
                     & 点估计     & 标准误    & p.value         & \multicolumn{1}{c}{备注}    \\ \hline
截距                   & -1.4102 & 0.4727 & \textbf{<0.01} &                           \\ 
treat-药物组            & 0.5182  & 0.2303 & \textbf{0.024}          & 以安慰剂组为基准组                 \\
weight-体重偏重          & -0.0518 & 0.2297 & 0.822          & 以体重偏轻为基准组                 \\
region-Europe        & 0.3865  & 0.2774 & 0.164          & \multirow{2}{*}{以Asia为基准组} \\
region-North America & 0.0089  & 0.3107 & 0.977          &                           \\
age                  & 0.0058  & 0.0088 & 0.506          &                          \\ \hline
\end{tabular}
\end{table}



\begin{table}[H]
	\centering
	\caption{辛普森悖论现象}
	\begin{tabular}{lccc}
		\hline
		& log(RR) point estimate & RR point estimate & log(RR) se \\ \hline
 全体     & 0.5182                 & 1.6790            & 0.2303     \\
 体重偏重亚组 & 0.6608                 & 1.9363            & 0.3729     \\
 体重偏轻亚组 & 0.4353                 & 1.5454            & 0.2929     \\ \hline
		& log(RR) 95\% CI        & RR 95\% CI        &            \\ \hline
 全体     & (\textbf{0.0669},0.9695)        & (\textbf{1.0691},2.6366)   &            \\
 体重偏重亚组 & (\textbf{-0.0702},1.3917)       & (\textbf{0.9322},4.0218)   &            \\
 体重偏轻亚组 & (\textbf{-0.1388},1.0094)       & (\textbf{0.8704},14.7055)  &           \\ \hline
 \end{tabular}
	\end{table}






\end{document}