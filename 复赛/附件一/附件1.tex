\documentclass{article}
\usepackage[UTF8]{ctex}
% Replace `letterpaper' with`a4paper' for UK/EU standard size
\usepackage[a4paper,top=2cm,bottom=2cm,left=3cm,right=3cm,marginparwidth=1.75cm]{geometry}

% Useful packages
\usepackage{amsmath}
\usepackage{graphicx}
\usepackage[colorlinks=true, allcolors=blue]{hyperref}
\usepackage{graphicx} %插入图片的宏包
\usepackage{float} %设置图片浮动位置的宏包
\usepackage{subfigure} %插入多图时用子图显示的宏包
\usepackage{parskip}
\usepackage{indentfirst} 
\setlength{\parindent}{2em}
\usepackage{hyperref}  
\usepackage{tikz}
\allowdisplaybreaks
\usepackage{multirow}
\usepackage{amsmath}
\usepackage{amsfonts,amssymb} 
\usepackage{xcolor} % 用于显示颜色
\usepackage{listings} % 用于插入代码
% \usepackage{tikz} % 导入多重比较示意图专用
% \usetikzlibrary{decorations,arrows,shapes} % 导入多重比较示意图专用

\lstset{
	basicstyle          =   \sffamily,          % 基本代码风格
	keywordstyle        =   \bfseries,          % 关键字风格
	commentstyle        =   \rmfamily\itshape,  % 注释的风格,斜体
	stringstyle         =   \ttfamily,  % 字符串风格
	flexiblecolumns,                % 别问为什么,加上这个
	numbers             =   left,   % 行号的位置在左边
	showspaces          =   false,  % 是否显示空格,显示了有点乱,所以不现实了
	numberstyle         =   \zihao{-5}\ttfamily,    % 行号的样式,小五号,tt等宽字体
	showstringspaces    =   false,
	captionpos          =   t,      % 这段代码的名字所呈现的位置,t指的是top上面
	frame               =   lrtb,   % 显示边框
}

\lstdefinestyle{R}{
	language        =   R, % 语言选R
	basicstyle      =   \zihao{-5}\ttfamily,
	numberstyle     =   \zihao{-5}\ttfamily,
	keywordstyle    =   \color{blue},
	keywordstyle    =   [2] \color{teal},
	stringstyle     =   \color{magenta},
	commentstyle    =   \color{red}\ttfamily,
	breaklines      =   true,   % 自动换行,建议不要写太长的行
	columns         =   fixed,  % 如果不加这一句,字间距就不固定,很丑,必须加
	basewidth       =   0.5em,
}

\lstdefinestyle{Python}{
	language        =   Python, % 语言选R
	basicstyle      =   \zihao{-5}\ttfamily,
	numberstyle     =   \zihao{-5}\ttfamily,
	keywordstyle    =   \color{blue},
	keywordstyle    =   [2] \color{teal},
	stringstyle     =   \color{magenta},
	commentstyle    =   \color{red}\ttfamily,
	breaklines      =   true,   % 自动换行,建议不要写太长的行
	columns         =   fixed,  % 如果不加这一句,字间距就不固定,很丑,必须加
	basewidth       =   0.5em,
}

\title{附件一:样本量重估的算法描述}
\author{}
\date{}
\begin{document}
\maketitle

\section{算法流程说明}
该算法输入为:试验前预计总样本量$N$,已完成的样本量$n$,试验允许的最大样本量$N_{\text{max}}$,
两组各自的应答率$\eta_0,\eta_1$,
总的单侧检验显著性$\alpha$(默认为0.025),把握度power$=1-\beta$(默认为0.95),
迭代中止标准$tol$(即条件把握度与期望把握度的最大容忍误差)。

该算法输出包括如下3种情况:
\begin{itemize}
    % \item 重估样本量小于等于$n+1$,即当前样本量已经足够,可以结束试验
    \item 重估样本量大于试验允许最大样本量$N_{\text{max}}$,把握度不足,建议结束试验。
    \item 给出介于$n$和$N_{\text{max}}$的重估样本量$N'$。
    \item $tol$设置过小,无法满足二分法迭代停止准则,需重新设置。
\end{itemize}

算法可以分为以下四个步骤:
\begin{enumerate}
    \item 设置最小样本量为$N_{\text{min}} = n$;
    \item 计算最大样本量$N_{\text{max}}$对应的条件把握度CP,
        若在最大样本量时CP<power,则结束试验,试验失败,无效中止。
    \item 取$N'=\lfloor (N_{\text{max}}+N_{\text{min}})/2\rfloor$,并计算CP。
        若$|\text{CP}-\text{power}|\le tol$则停止迭代,当前$N'$即为重估后的样本量,
        否则,进入第四步。
    \item 若$\text{CP}>\text{power}$说明当前样本量偏大,设置$N_{\text{max}}=N'$;
        若$\text{CP}<\text{power}$说明当前样本量偏小,设置$N_{\text{min}}=N'$,重复第三步。
\end{enumerate}

\section{算法统计原理简介}
根据陈建平等人的研究\cite{陈建平2010期中分析的条件把握度及样本含量再估计},
条件把握度CP的计算公式为:
\begin{align*}
    \text{CP}  = 1-\Phi\left[\frac{Z_{1-\alpha^*}-[B(\tau)+(1-\tau)\theta]}{\sqrt{1-\tau}} \right] 
\end{align*}

其中,$\alpha^*$是用消耗函数调整后的单侧显著性水平,这里采用较为保守的O'Brien-Fleming消耗函数
\[\alpha(\tau)=1-\Phi(\frac{Z_{1-\alpha}}{\sqrt{\tau}})\]

$B(\tau)=z(\tau)\sqrt{\tau}$是我们构造的服从于布朗运动的统计量,方差为$\tau$。
$z(\tau)$是在期中检验时,两组应答率之差的z-score标准化统计量:
\[z(\tau)=\frac{\eta_1-\eta_0}{\sqrt{\frac{\eta_1(1-\eta_1)}{n/2} + \frac{\eta_0(1-\eta_0)}{n/2}}}\]
$z(\tau)$渐进服从标准正态分布。

$\theta$是指按原计划样本量完成试验时,所计算得到的统计量。期中分析时的估计值为
\[\hat{\theta} = \frac{B(\tau)}{\tau}\]

那么,结合上述各式,可以得到条件把握度的计算公式为:
\[\text{CP}=1-\Phi\left[\frac{Z_{1-\alpha^*}-[z(\tau)\sqrt{\tau}+(1-\tau)z(\tau)/\sqrt{\tau}]}
										{\sqrt{1-\tau}}\right] \]
其中,信息时间$\tau=\frac{n}{N}$。







\bibliographystyle{plain}
\bibliography{ref}

\end{document}


% \begin{figure}[H]
%     \centering  %图片全局居中
%     \includegraphics[width=0.8\textwidth]{result3}
%     \caption{The heap after performing INCREASE-KEY(A, 10, 28)}
% \end{figure}